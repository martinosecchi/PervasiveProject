\documentclass{sigchi}

% Arabic page numbers for submission. 
% Remove this line to eliminate page numbers for the camera ready copy
%\pagenumbering{arabic}


% Load basic packages
\usepackage{balance}  % to better equalize the last page
\usepackage{graphics} % for EPS, load graphicx instead
\usepackage{times}    % comment if you want LaTeX's default font
\usepackage{url}      % llt: nicely formatted URLs
\usepackage{color}

% llt: Define a global style for URLs, rather that the default one
\makeatletter
\def\url@leostyle{%
  \@ifundefined{selectfont}{\def\UrlFont{\sf}}{\def\UrlFont{\small\bf\ttfamily}}}
\makeatother
\urlstyle{leo}


% To make various LaTeX processors do the right thing with page size.
\def\pprw{8.5in}
\def\pprh{11in}
\special{papersize=\pprw,\pprh}
\setlength{\paperwidth}{\pprw}
\setlength{\paperheight}{\pprh}
\setlength{\pdfpagewidth}{\pprw}
\setlength{\pdfpageheight}{\pprh}

% Make sure hyperref comes last of your loaded packages, 
% to give it a fighting chance of not being over-written, 
% since its job is to redefine many LaTeX commands.
\usepackage[pdftex]{hyperref}
\hypersetup{
pdftitle={report},
pdfauthor={LaTeX},
pdfkeywords={SIGCHI, proceedings, archival format},
bookmarksnumbered,
pdfstartview={FitH},
colorlinks,
citecolor=black,
filecolor=black,
linkcolor=black,
urlcolor=black,
breaklinks=true,
}

% create a shortcut to typeset table headings
\newcommand\tabhead[1]{\small\textbf{#1}}
\newcommand\todo[1]{\textcolor{red}{#1}}


% End of preamble. Here it comes the document.
\begin{document}

\title{Smart Bin - An Odor Oriented Approach}

\numberofauthors{2}
\author{
  \alignauthor Ivan Naumovski\\
    \email{inau@itu.dk}\\
  \alignauthor Martino Secchi\\
    \email{msec@itu.dk}\\
}

\maketitle

\begin{abstract}
Technology enhanced trash cans have already been subject to research and have become available as market products.
How we handle the waste has traditionally been a logistics issue, but it can be approached also in other ways. The SmartBin perceives the trash as something more than a pile of waste - it is also something smelly!
The emphasis is on improving the indoor environment to ultimately improve the quality of life by detecting odors.
The technology enhancing the bin uses sensors which detect gas emissions, mainly the ones occuring in decomposition of organic materials.
If any of these values exceed the threshold the SmartBin will react accordingly, supported by state of the art machine learning techniques.
Any specifics will be listed in the following document. This will range from hardware prototyping to evaluation of the product.
\end{abstract}

\category{H.5.m.}{Pervasive computing, smart measuring}{Miscellaneous}


\terms{
	Design; Measurement. 
}



\section{Introduction}
Waste management is a constant issue, that is dealt with on the daily by different organisations.
According to Eurostat, the countries in europe have had a rather stable production of waste the past decade, while some countries fluctuate towards higher productions of waste other countries produce less. \todo{citation needed}
The above supports the claim that waste management is highly relevant in a post modern society.

However waste management has traditionally been perceived from a logistical point of view.
Imagine not the amounts nor the location of the bin being the driving force for interacting with it, but rather its smell.

This is the essential feature of our system - the additional parameter.
This additonal information can be used in conjunction with existing systems or be used to produce an entirely new experience centered around the waste bin.
Following this concept, a prototype of a simple system has been developed.
The system produces information about smell and trash level for the bin entity it is installed on.
This information is made readily available using a cloud-based system. 
We have one android application making use of the data. Its designed to be able to present in two different modes, namely as an ambient display or as an smartphone app.
Figure ~\ref{fig:smartbin} shows an overview of the system.

The system is specifically designed for private use inside a household, but can be easily adjusted and extended for different usages.

In this paper we describe the system specifications and evaluation, and discuss its potential integration in other scenarios different from the one presented.

\begin{figure}
\centering
\includegraphics[scale=.2]{img/smartbin}
\includegraphics[scale=.075]{img/IMG-20161130-WA0000}
\caption{The SmartBin system overview, the bin with ambient display}
\label{fig:smartbin}
\end{figure}

\section{Related Work}
This is not the first paper about Smart Waste Management Systems (WMS).
Other researches explored interesting IoT approaches to the problem, mostly in relation to planning garbage collection and / or waste reduction.
\todo{...many, pick one or two... like in Australia, France} implemented an RFID and weight  based approach for a real time automated WMS, with the main focus on bringing down management costs and facilitate automating waste identification.
\todo{ citation to korean guys } In another study from South Korea, the main approach was to identify food waste in a selected area of Seoul and give citizens incentive to waste less food by fining them based on the amounts of waste they dispose.
The infrastructure is similar to other systems, with a centralized server and a host of devices providing data to this server.
Then the server provides data for applications such as management utilities or phone apps.
\todo{ cite catania} In another study, Vincenzo Catania and his colleagues used a Smart-M3 Platform and sensor enhanced bins in Catania, Italy with the main focus on urban planning, smart collection and  monitoring of urban solid waste. In this case, the information that was collected was on the location of the trash can, level of fullness, and weight of the waste.

\section{System Design}
In this section we describe the overall system characteristics proposed for our approach to waste management.\\
The system is composed of two main components:
\begin{itemize}
\item sensory layer: responsible of producing information.
\item information layer: responsible of collecting and presenting information.
\end{itemize}

\subsection{sensory layer}
Our approach relies on two kinds of information related to the state of the bin: smell and fill level.
For this reason, the sensory part of the system is of key importance, since it enables the information to be perceived and registered by the system.
An odor detection sensor is able to capture volatile organic compounds (VOCs) and other gases, typical bioproducts of food decomposition, commonly associated with bad smell. 
Some examples are Hydrogen Sulphide, Ammonia, Toluene and others.
When the presence of one of these gases exceeds a given boundary, the sensor will report bad smell.
The second sensor will be placed on the top of the trash can, in order to measure the distance to the trash level and report about fullness.

\subsection{information layer}
Having obtained the information is only the first step though, the next one is to make it available.
Our system uses a web service for collecting the data and storing it in a database, making it accessible to potential consumers.

\section{Hardware Architecture}
The Smartbin hardware platform is a simple construction consisting of two sensors, a microprocessor and a wifi communication module. It is embedded into a thrashbin with a lid which is used
as a base for both the distance sensor and the gas sensor.

We built the prototype using an Arduino Uno board.
 As for the sensors we used a Figaro TGS2602 as our gas sensor and a MaxSonar MB1013 ultrasonic rangefinder as our distance sensor.
We use a WiFi shield on the Arduino board for wireless communication.

The presentation layer works on Android 4.0 or later.

\section{Software Architecture}
The software system consists of three components. The module on the bin, one cloud service and one android client.
The components interact using the client-server model over the web, in a cloud-based architecture (figure ~\ref{fig:architecture}).

\begin{figure}
\centering
\includegraphics[scale=.6]{img/architecture}
\caption{Cloud-based architecture}
\label{fig:architecture}
\end{figure}

\begin{figure}
\centering
\includegraphics[scale=.3]{img/screen_admin}
\includegraphics[scale=.3]{img/screen_single}
\includegraphics[scale=.3]{img/screen_stats}
\caption{Image of the three main views in the android client} 	
\label{fig:clientmodes}
\end{figure}

\begin{figure}
\centering
\includegraphics{img/app_colors_nb}
\caption{Image of the color codes ranging from deviation under the norm to deviation over the norm} 
\label{fig:colorcodes}
\end{figure}

\textit{ \textbf{On the bin}}, the gas sensor first needs to go through a warm-up period of about 15-20 minutes, during which the sensor is energised. The resistance registered by the sensor drops sharply for the first minutes after energising, regardless of the presence of gases. This behaviour is commonly known as \textit{Initial Action}, and could cause a system to mistakenly report presence of gases during the first minutes of activity\cite{tgs2602}.

After the warm-up period, the system will go through a period of \textit{calibration} (5 minutes), during which the value Ro (resistance measured by the gas sensor in clean air) is computed as an average of 300 samples.
During this period the sensor has to be kept in "clean air", or  in an environment considered to be in a base condition, because later measurements will rely on this value.

After warm up and calibration, the gas concentration sent by the module will be in the form of a ratio \textit{Rs/Ro}, where Rs is the resistance measured by the sensor at a given time and Ro is the constant value detected in clean air.
This means that if the gas concentrations perceived by the sensor are very similar as the ones in clean air, the ratio will be close to 1.
The more the values Rs and Ro differ, the more far from 1 the ratio will be.

\textit{\textbf{The cloud service}} is hosted using Google App Engine, it is a simple website that can present the state of SmartBins in the system. It also exposes a REST style API - which can provide SmartBins and Contexts.
It uses technologies such as HTML, JavaScript, Java Servlets and schemeless persistent storage.

\textit{\textbf{The android client}} can provide statistics about different smartbins - it only supports  presentation of data in its current form.
It supports two modes: \textbf{Admin mode} and \textbf{Single Bin mode}.
Admin mode includes an overview of every single bin and graphs for each one. The graphs are only available as long as enough context information is present see figure~\ref{fig:clientmodes}.

 Single bin mode provides a view of the current bin state.
 
 The bin state is divided into five different categories, this is represented visually as colors.
 The order is, \textit{blue}, \textit{dark green}, \textit{green}, \textit{orange}, and \textit{red}, as seen on figure~\ref{fig:colorcodes}.
The middle one, \textit{green}, is when the concentration is deviating with at most 5\%, next tier \textit{dark green} and \textit{orange} is a deviation of at most 25\%, and the last ones \textit{blue} and \textit{red} are deviations above 25\%.



\section{Evaluation in Smelly Environments}
The SmartBin system which has been constructed utilizes gas sensors to open up an additional dimension when measuring the state of the environment, in which it has been placed.

This extra dimension provides values of various gas concentrations in the current environment.
In short these values will be used in an evaluation of how the current environment is doing in regards to air quality and potential safety hazards for individuals.

The expectation is that these results can be used to produce rather precise predictions about how smelly the air in the given environment is.

People have a notion about what smells, machines need to do these predictions based on values from sensors.

The assumption is that the gas emissions of some chemical reactions, namely the ones where decomposition of organic materials is happening, always produce some specific gasses as a product.
It is based on our knowledge of these different gasses and how they smell - we assume that hightened concentrations of some of these gas types result in worse quality air.


\subsection{System Evaluation}

We interviewed a restricted poll of people in regards to usability of our system in a private scenario, and the results are the following:
\todo{actual findings?}

all of the interviewed keep their private trash under the sink in a closed compartment

most of the interviewed consistently forget to take out the trash when they should

most of the interviewed think a reminder on the trash status placed on the door would be extremely useful to remember to take the trash out

evrybody think it's cool

stuff


\section{Conclusion}

The system described in this paper is a simple implementation of what waste management system can look like with the use of odor detection. 
It is designed for private usage inside the household, as one of the many different features that can turn a normal home into a smart home.

With the right IoT components, trash status can be easily monitored and transmitted, all with \textit{off-the-shelf} and relatively cheap components.

The main purpose of the system is to provide the means for monitoring trash conditions inside the house, and to act as a reminder to the users so that they can dispose of their waste before such conditions become problematic.

It's been discussed how taking smell into account for waste management could lead to better air and sanitary conditions around the trash.

Even existing systems could be adjusted to take into account the smell factor, all with relative ease and at a low price, just by adding a small sensor to their infrastructure.




\bibliographystyle{acm-sigchi}
\bibliography{ubicomp}
\end{document}
