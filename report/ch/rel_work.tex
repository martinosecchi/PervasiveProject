Smart Waste Management Systems (WMS) have been explored before. The usual approach is to explore the physical attributes of a bin, namely what is put in the bin.
This information is usually used to perform advanced planning for collection of waste, or influencing people to adjust their behaviour in regards to what to throw out.

Some research groups implemented RFID and weight  based systems for real time automated WMS, with the main focus on bringing down management costs and facilitate automating waste identification~\cite{france}~\cite{australia}.

In another study performed in South Korea, the main approach was to identify food waste in a selected area of Seoul and give citizens incentive to waste less food by fining them based on the amounts of waste they dispose~\cite{korea}.

In an italian study, performed by Vincenzo Catania and his colleagues in the city of Catania, a Smart-M3 Platform and sensor enhanced bins were used, this was done with the main focus on urban planning, smart collection and  monitoring of urban solid waste \cite{catania}.\\
In this case, the information that was collected was on the location of the trash can, level of fullness, and weight of the waste.

The infrastructure used in the variety of systems tends to be similar.
Usual approach is a centralized server and a host of devices providing data to the server.\\
The server is then used as a data provider for powering applications such as management utilities or phone apps.