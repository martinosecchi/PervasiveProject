Smart Waste Management Systems (WMS) have been subject to extended research throughout the years.
The usual approach for data collecting in this area is to explore the physical attributes of a bin, namely what is put in the bin.
The common ways of obtaining such information are divided in two main categories: physical sensing and tagging.

Physical sensing is a great way to obtain generic information about the amount of garbage, and is usually measured in weight or height.
In recent years though, the use of tagging has become fairly popular, and some research in WMS has been conducted through the use of RFID tags and other tagging technologies.

With different technologies, in general come also different purposes.
While both approaches can handle well general monitoring and analysis of the processes revolving around waste management, some are better suited for specific uses.

Tagging technologies are excellent ways of tracking every object and associate meta information to them, therefore are often used in conjunction with recycling systems.
On the other hand, amount information is often used for systems that focus on resource allocation, garbage collection or process planning.

Systems with mixed approaches have also been developed.
%This information is traditionally used to perform advanced planning for collection of waste, or influencing people to adjust their behaviour in regards to what to throw out.

There are numerous examples of research in this area, we mention just few examples in the following paragraphs.

In a study conducted in Australia, a research group implemented an RFID and weight based system for real time automated WMS, with the main focus on bringing down management costs and facilitate automating waste identification~\cite{australia}.
In France, an RFID based system is instead focused in assisting in the recycling process by making sure that the waste is disposed correctly\cite{france}.

In another study performed in South Korea, the main approach was to analyse and identify food waste in a selected area of Seoul, and give citizens incentive to waste less food by fining them based on the amounts of waste they dispose~\cite{korea}. The distinction of trash was done using RFID, and the amount was measured in weight.

In an italian study, a Smart-M3 Platform and sensor enhanced bins were used, this was done with the main focus on urban planning, smart collection and  monitoring of urban solid waste \cite{catania}.\\
In this case, the information that was collected was on the location of the trash can, level of fullness, and weight of the waste.


It's worth mentioning that all of the previous examples don't account for gas emission or smell, even though some steps in this direction have been taken.
A research at MIT (Massachusetts Institute of Technology) enabled a trash can with a perfume dispenser, that would activate whenever gas emissions are detected from the trash\cite{perfume}.


%The infrastructure used in the variety of systems tends to be similar.
%Usual approach is a centralized server and a host of devices providing data to the server.\\
%The server is then used as a data provider for powering applications such as management utilities or phone apps.