Waste management is a constant issue, that is dealt with on the daily by different organisations.
According to Eurostat, the countries in europe have had a rather stable production of waste the past decade, while some countries fluctuate towards higher productions of waste other countries produce less. \todo{citation needed}
The above supports the claim that waste management is highly relevant in a post modern society.

However waste management has traditionally been perceived from a logistical point of view.
Imagine not the amounts nor the location of the bin being the driving force for interacting with it, but rather its smell.

This is the essential feature of our system - the additional parameter, which can be used in conjunction with existing systems or used to produce entirely new experiences centered around the waste bin.

Following this concept, we developed a prototype of a simple system that produces information about smell and trash level, and pairs this information with a specific bin where the device is installed. 

The system is specifically designed for private use inside a household, but can be easily adjusted and extended for different usages.

In this paper we describe the system specifications and evaluation, and discuss its potential integration in different scenarios than the one presented.