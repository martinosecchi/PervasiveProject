
In this section we describe the overall system characteristics proposed for our approach to waste management.\\
<<<<<<< HEAD
The SmartBin system is composed of two main components:
=======
The system is composed of four main components:
>>>>>>> 61ee2b311c060072ab45edb0cf99b7b57401e828
\begin{itemize}
\item sensory layer: responsible of producing information, placed directly on the bin.
\item communication module: responsible of sending such information.
\item data layer (the cloud): this component stores the data and makes it available.
\item data consumer (android):  
\end{itemize}

<<<<<<< HEAD
\subsection{sensory layer}
The sensory layer has two sensors and one communication device. The sensors detect the state of the bin.
Our approach relies on two kinds of information related to the state of the bin: smell and fill level.
For this reason, the sensory part of the system is of key importance, since it enables the information to be perceived and registered by the system.
An odor detection sensor is able to capture volatile organic compounds (VOCs) and other gases, typical bioproducts of food decomposition, commonly associated with bad smell. 
Some examples are Hydrogen Sulphide, Ammonia, Toluene and others.
The smartbin sends the raw values of VOC concentration to the information layer.

The other type of sensor is a ultrasonic distance measurement unit. This is used to get the distance to the bottom of the bin.
This type of information is processed in the sensory layer to reflect the amount of thrash filled in the bin.
=======
\subsection{Sensory layer}
Our approach relies on two kinds of information related to the state of the bin: smell and fill level.
For this reason, the sensory part of the system is of key importance, since it enables the information to be perceived and registered by the system.
An odor detection sensor is able to capture volatile organic compounds (VOCs) and other gases, typical bioproducts of food decomposition, commonly associated with bad smell. 
Some examples are Hydrogen Sulphide (\ce{H_{2}S}), Ammonia(\ce{NH_3}), Toluene (\ce{CH_3}) and others.
When the presence of one of these gases exceeds a given boundary, the sensor will report bad smell.
The second sensor will be placed on the top of the trash can, in order to measure the distance to the trash level and report about fullness.
>>>>>>> 61ee2b311c060072ab45edb0cf99b7b57401e828

\subsection{Communication and Data layer}
Having obtained the information is only the first step though, the next one is to make it available.
<<<<<<< HEAD
Our system uses a web service for collecting the data and storing it in a database, making it accessible to potential consumers.
The service supports two types of entities. The SmartBins and the contexts.
In short a SmartBin represent the hardware instance of a smartbin, this includes information such as location expressed as coordinates, unique identifier, 'calibration' which is the intial air quality level(raw value of VOCs when empty) and meta information about the last associated context.
This ensures that status of every bin is readily available without having to traverse the contexts.

The contexts model a snapshot of a bins state at a given time.
=======
The data are sent directly from the bin to the cloud. In our implementation this is done via WiFi, but can ideally be sent in any other way.
Our system uses a cloud-based web service for collecting the data and storing it in a database, making it accessible to potential consumers.

\subsection{Data Consumer}
Finally, the information is accessed and presented on an Android application, and made accessible to everyone. 
>>>>>>> 61ee2b311c060072ab45edb0cf99b7b57401e828
