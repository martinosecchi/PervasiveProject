
In this section we describe the overall system characteristics proposed for our approach to waste management.\\
The system is composed of four main components:
\begin{itemize}
\item sensory layer: responsible of producing information, placed directly on the bin.
\item communication module: responsible of sending such information.
\item data layer (the cloud): this component stores the data and makes it available.
\item data consumer (android):  
\end{itemize}

\subsection{Sensory layer}
Our approach relies on two kinds of information related to the state of the bin: smell and fill level.
For this reason, the sensory part of the system is of key importance, since it enables the information to be perceived and registered by the system.
An odor detection sensor is able to capture volatile organic compounds (VOCs) and other gases, typical bioproducts of food decomposition, commonly associated with bad smell. 
Some examples are Hydrogen Sulphide (\ce{H_{2}S}), Ammonia(\ce{NH_3}), Toluene (\ce{CH_3}) and others.
When the presence of one of these gases exceeds a given boundary, the sensor will report bad smell.
The second sensor will be placed on the top of the trash can, in order to measure the distance to the trash level and report about fullness.

\subsection{Communication and Data layer}
Having obtained the information is only the first step though, the next one is to make it available.
The data are sent directly from the bin to the cloud. In our implementation this is done via WiFi, but can ideally be sent in any other way.
Our system uses a cloud-based web service for collecting the data and storing it in a database, making it accessible to potential consumers.

\subsection{Data Consumer}
Finally, the information is accessed and presented on an Android application, and made accessible to everyone. 