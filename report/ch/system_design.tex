
In this section we describe the overall system characteristics proposed for our approach to waste management.\\

The system is composed of four main components:
\begin{itemize}
\item sensory layer: responsible of producing information, placed directly on the bin.
\item communication module: responsible of sending such information.
\item data layer (the cloud): this component stores the data and makes it available.
\item data consumer (android):  this client shows information to the end user on request
\end{itemize}

\subsection{Sensory layer}
The sensory layer has two sensors and one communication device. The sensors detect the state of the bin.
Our approach relies on two kinds of information related to the state of the bin: smell and fill level.
For this reason, the sensory part of the system is of key importance, since it enables the information to be perceived and registered by the system.
An odor detection sensor is able to capture volatile organic compounds (VOCs) and other gases, typical bioproducts of food decomposition, commonly associated with bad smell. 
Some examples are Hydrogen Sulphide (\ce{H_{2}S}), Ammonia(\ce{NH_3}), Toluene (\ce{CH_3}) and others.
The smartbin sends the raw values of VOC concentration to the information layer.

The other type of sensor is a ultrasonic distance measurement unit. This is used to get the distance to the bottom of the bin.
This type of information is processed in the sensory layer to reflect the amount of thrash filled in the bin.

\subsection{Communication and Data layer}
Having obtained the information is only the first step though, the next one is to make it available.

Our system uses a web service for collecting the data and storing it in a database, making it accessible to potential consumers.
The service supports two types of entities. The SmartBins and the contexts.
In short a SmartBin represents the hardware instance of a smartbin, this includes information such as location expressed as coordinates, unique identifier, 'calibration' which is the intial air quality level (in "clean" air, a value to which compare later measurements) and meta information about the last associated context.
This ensures that status of every bin is readily available without having to traverse the contexts.

The contexts model a snapshot of a bins state at a given time.

The data are sent directly from the bin to the cloud. In our implementation this is done via WiFi, but can ideally be sent in any other way.
Our system uses a cloud-based web service for collecting the data and storing it in a database, making it accessible to potential consumers.

\subsection{Data Consumer}
Finally, the information is accessed and presented on an Android application, and made accessible to the users. 
The client can work in two separate modes: as an android application or simply as an ambient display.

On the android application it is possible to manage multiple bins and get an overview of the last values measured by the system.

The ambient display is much more simple, it just displays a color of a shade from green to red depending on the scent level. 

In our system we use an android phone to run in both modes, so pressing on the screen even in ambient display mode will give the user the extra functionality of knowing trash level and emission level.

Ideally the ambient display can be placed in a strategic location inside the house, for example by the door or on the fridge, where people can be reminded of the trash status and act accordingly.