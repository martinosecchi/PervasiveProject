\iffalse
discussion

here emphasize  about the famous What IF question

take this INNOVATIVE aspect and use it in waste management

in the conclusion:
we can build it, it works, the technology is out there
in the discussion:
it can be applied somewhere else
it can extend other systems
used for innovation
it's a different take
\fi

The SmartBin system which has been constructed utilises gas sensors to open up an additional dimension when measuring the status of the trash present in bins.
This extra dimension is the main focus of our system, as an innovative angle on an existing problem.

In this section we want to discuss some of the openings that this approach allows.

While we have tested this approach in small scale in a home environment, it is not limited to this setting.\\
What if we could use this data to expand our take on waste management systems?

Logistic for garbage collection is usually driven by amounts, but what if bad smell was also one of the leading parameters to act upon?

Large levels of bad smell emitting from the trash result in unsanitary conditions, insalubrity and bad air quality, which can attract insects and other wild animals.
While it remains true that trash handling is first and foremost a logistical issue, most waste management systems don't take bad smell into account at all.

By prioritising the collection of "smelly" garbage, alongside with the collection of full trash cans, insalubrity conditions might be reduced significantly.
This would ultimately improve air quality and the quality of life around public spaces and all this can be achieved with relative ease.\\
Other smartbin systems already have means of communication integrated, so adding an additional sensor should not be highly problematic.

The components we have used for our prototype might not be the optimal components for systems in public spaces.
The biggest weakness is the power consumption that WiFi modules have. Alternatives could be technologies like LoRa that are designed for lower power consumption\ref{lora}.
 

\iffalse
%old evaluation
The SmartBin system which has been constructed utilises gas sensors to open up an additional dimension when measuring the state of the environment in which it has been placed.
This extra dimension provides values of various gas concentrations in the current environment.
In short these values will be used in an evaluation of how the current environment is doing in regards to air quality and potential insalubrity hazards for individuals.
The expectation is that these results can be used to produce rather precise predictions about how smelly the air in the given environment is.
People have a notion about what smells, machines need to do these predictions based on values from sensors.
The assumption is that the gas emissions of some chemical reactions, namely the ones where decomposition of organic materials is happening, always produce some specific gases as a product.
It is based on our knowledge of these different gasses and how they smell - we assume that hightened concentrations of some of these gas types result in worse quality air.


%old conclusion
While it remains true that trash handling is first and foremost a logistical issue, most waste management systems don't take bad smell into account.
Our claim is that bad smell is actually a part of the problem when talking about trash management,  both in the private as well as in the public.

While a private IoT "smart bin" that includes smell information would serve the ultimate goal of assisting in the process of garbage collection, it can be extended also for other purposes like data collection or as a perfume dispenser \cite{perfume}.
On the other hand, when approaching waste management from a urban point of view, the focus is often placed on the logistics: urban planning (where to place the bins, how many, ..) and on route planning for garbage collection.
These systems usually rely on information on amounts of trash, being it height values or weight.
Our claim \todo{not supported by proof} is that smell information could also be relevant.
\todo{rephrase or change, too tired right now}
\fi