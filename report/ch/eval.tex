The SmartBin system which has been constructed utilizes gas sensors to open up an additional dimension when measuring the state of the environment, in which it has been placed.

This extra dimension provides values of various gas concentrations in the current environment.
In short these values will be used in an evaluation of how the current environment is doing in regards to air quality and potential safety hazards for individuals.

The expectation is that these results can be used to produce rather precise predictions about how smelly the air in the given environment is.

People have a notion about what smells, machines need to do these predictions based on values from sensors.

The assumption is that the gas emissions of some chemical reactions, namely the ones where decomposition of organic materials is happening, always produce some specific gasses as a product.
It is based on our knowledge of these different gasses and how they smell - we assume that hightened concentrations of some of these gas types result in worse quality air.


\subsection{System Evaluation}

We interviewed a restricted poll of people in regards to usability of our system in a private scenario, and the results are the following:
\todo{actual findings?}

Multiple groups were used.

Group A where a homogenous group of people, all males and all ranging between the age of 20-30.
They keep their thrash bin below their sinks in the kitchen, and some also have additional bins in the bathroom.
The bathroom bins were not deemed that relevant due to the fact that most bathrooms are subject to having smelly environments post use.
Interviews collecting the opinion of the group  were performed.
The procedure was to present images of the bin and let the participants play around with the android application, followed by a non-chalant conversation of the system.
All three people expressed that the system seemed like something obsolete. They were all really good at taking out the thrash.
And the idea of having a situated display on either the wall or the outside of the thrash cabinet was stupid due to the fact that one could just open up and see.

all of the interviewed keep their private trash under the sink in a closed compartment

most of the interviewed consistently forget to take out the trash when they should

most of the interviewed think a reminder on the trash status placed on the door would be extremely useful to remember to take the trash out

evrybody think it's cool

stuff
