
We interviewed a restricted poll of people in regards to usability of our system in a private scenario, 
in two different occasions.

We used two distinct focus groups, and we will refer to them as group A and group B in this paper.

Group A was a homogenous group of people, all males and all ranging between the age of 24-30, all with IT-related background.
Group B was instead a more diverse poll of people, mostly females of age between 24-30, and non IT-related background.

\subsection{Group A}

Group A has been presented with the android app and some pictures of the bin and its setup.
During the follow-up discussion, the general reaction from the group was that the system felt unnecessary and was not welcomed. 
The purpose of the system to serve as a reminder and state monitoring device was easily achieved "manually" by the participants, who regularly took the trash out and checked its status in person.

\subsection{Group B}

Group B was introduced to the system in loco and in its totality, with bin, situated display and android app, all embedded in the house environment. 
The reactions from this focus group were much more positive, especially towards the situated display placed on the wall.
The general sentiment was of marvel and appreciation of the achievements of technology even in the ordinary life scale.
Regardless of the actual purpose of the system, the prototype was mostly welcomed as an instance of what a larger system could become in the future of smart houses.
That said, the fact that bad smell could actually be measured by machines was a surprise to most, and the visual reminder of the trash status right by the door was most appreciated by the ones that regularly forget to take out the trash.
