The system described above is a simple implementation of what a private waste management system can look like. 
With the right IoT components, trash status can be easily monitored and transmitted.
Modern sensors allow machines to perceive the same as what humans can, if not more at times, making it possible to even detect bad smell in our trash.
Here is where our system separates itself from most of the other products in the same area.
This extra parameter could be exploited to potentially be one of the leading information considered in waste management, together with the more canonical weight and amount. 

While it remains true that trash handling is first and foremost a logistical issue, most waste management systems don't take bad smell into account.
Our claim is that bad smell is actually a part of the problem when talking about trash management,  both in the private as well as in the public.

While a private IoT "smart bin" that includes smell information would serve the ultimate goal of assisting in the process of garbage collection, it can be extended also for other purposes like data collection or as a perfume dispenser \cite{perfume}.
On the other hand, when approaching waste management from a urban point of view, the focus is often placed on the logistics: urban planning (where to place the bins, how many, ..) and on route planning for garbage collection.
These systems usually rely on information on amounts of trash, being it height values or weight.
Our claim \todo{not supported by proof} is that smell information could also be relevant.
\todo{rephrase or change, too tired right now}